% Options for packages loaded elsewhere
\PassOptionsToPackage{unicode}{hyperref}
\PassOptionsToPackage{hyphens}{url}
%
\documentclass[
]{article}
\usepackage{lmodern}
\usepackage{amssymb,amsmath}
\usepackage{ifxetex,ifluatex}
\ifnum 0\ifxetex 1\fi\ifluatex 1\fi=0 % if pdftex
  \usepackage[T1]{fontenc}
  \usepackage[utf8]{inputenc}
  \usepackage{textcomp} % provide euro and other symbols
\else % if luatex or xetex
  \usepackage{unicode-math}
  \defaultfontfeatures{Scale=MatchLowercase}
  \defaultfontfeatures[\rmfamily]{Ligatures=TeX,Scale=1}
\fi
% Use upquote if available, for straight quotes in verbatim environments
\IfFileExists{upquote.sty}{\usepackage{upquote}}{}
\IfFileExists{microtype.sty}{% use microtype if available
  \usepackage[]{microtype}
  \UseMicrotypeSet[protrusion]{basicmath} % disable protrusion for tt fonts
}{}
\makeatletter
\@ifundefined{KOMAClassName}{% if non-KOMA class
  \IfFileExists{parskip.sty}{%
    \usepackage{parskip}
  }{% else
    \setlength{\parindent}{0pt}
    \setlength{\parskip}{6pt plus 2pt minus 1pt}}
}{% if KOMA class
  \KOMAoptions{parskip=half}}
\makeatother
\usepackage{xcolor}
\IfFileExists{xurl.sty}{\usepackage{xurl}}{} % add URL line breaks if available
\IfFileExists{bookmark.sty}{\usepackage{bookmark}}{\usepackage{hyperref}}
\hypersetup{
  pdftitle={Ant colony},
  pdfauthor={Jan Lennartz \& Andrei Chirita},
  hidelinks,
  pdfcreator={LaTeX via pandoc}}
\urlstyle{same} % disable monospaced font for URLs
\usepackage[margin=1in]{geometry}
\usepackage{color}
\usepackage{fancyvrb}
\newcommand{\VerbBar}{|}
\newcommand{\VERB}{\Verb[commandchars=\\\{\}]}
\DefineVerbatimEnvironment{Highlighting}{Verbatim}{commandchars=\\\{\}}
% Add ',fontsize=\small' for more characters per line
\usepackage{framed}
\definecolor{shadecolor}{RGB}{248,248,248}
\newenvironment{Shaded}{\begin{snugshade}}{\end{snugshade}}
\newcommand{\AlertTok}[1]{\textcolor[rgb]{0.94,0.16,0.16}{#1}}
\newcommand{\AnnotationTok}[1]{\textcolor[rgb]{0.56,0.35,0.01}{\textbf{\textit{#1}}}}
\newcommand{\AttributeTok}[1]{\textcolor[rgb]{0.77,0.63,0.00}{#1}}
\newcommand{\BaseNTok}[1]{\textcolor[rgb]{0.00,0.00,0.81}{#1}}
\newcommand{\BuiltInTok}[1]{#1}
\newcommand{\CharTok}[1]{\textcolor[rgb]{0.31,0.60,0.02}{#1}}
\newcommand{\CommentTok}[1]{\textcolor[rgb]{0.56,0.35,0.01}{\textit{#1}}}
\newcommand{\CommentVarTok}[1]{\textcolor[rgb]{0.56,0.35,0.01}{\textbf{\textit{#1}}}}
\newcommand{\ConstantTok}[1]{\textcolor[rgb]{0.00,0.00,0.00}{#1}}
\newcommand{\ControlFlowTok}[1]{\textcolor[rgb]{0.13,0.29,0.53}{\textbf{#1}}}
\newcommand{\DataTypeTok}[1]{\textcolor[rgb]{0.13,0.29,0.53}{#1}}
\newcommand{\DecValTok}[1]{\textcolor[rgb]{0.00,0.00,0.81}{#1}}
\newcommand{\DocumentationTok}[1]{\textcolor[rgb]{0.56,0.35,0.01}{\textbf{\textit{#1}}}}
\newcommand{\ErrorTok}[1]{\textcolor[rgb]{0.64,0.00,0.00}{\textbf{#1}}}
\newcommand{\ExtensionTok}[1]{#1}
\newcommand{\FloatTok}[1]{\textcolor[rgb]{0.00,0.00,0.81}{#1}}
\newcommand{\FunctionTok}[1]{\textcolor[rgb]{0.00,0.00,0.00}{#1}}
\newcommand{\ImportTok}[1]{#1}
\newcommand{\InformationTok}[1]{\textcolor[rgb]{0.56,0.35,0.01}{\textbf{\textit{#1}}}}
\newcommand{\KeywordTok}[1]{\textcolor[rgb]{0.13,0.29,0.53}{\textbf{#1}}}
\newcommand{\NormalTok}[1]{#1}
\newcommand{\OperatorTok}[1]{\textcolor[rgb]{0.81,0.36,0.00}{\textbf{#1}}}
\newcommand{\OtherTok}[1]{\textcolor[rgb]{0.56,0.35,0.01}{#1}}
\newcommand{\PreprocessorTok}[1]{\textcolor[rgb]{0.56,0.35,0.01}{\textit{#1}}}
\newcommand{\RegionMarkerTok}[1]{#1}
\newcommand{\SpecialCharTok}[1]{\textcolor[rgb]{0.00,0.00,0.00}{#1}}
\newcommand{\SpecialStringTok}[1]{\textcolor[rgb]{0.31,0.60,0.02}{#1}}
\newcommand{\StringTok}[1]{\textcolor[rgb]{0.31,0.60,0.02}{#1}}
\newcommand{\VariableTok}[1]{\textcolor[rgb]{0.00,0.00,0.00}{#1}}
\newcommand{\VerbatimStringTok}[1]{\textcolor[rgb]{0.31,0.60,0.02}{#1}}
\newcommand{\WarningTok}[1]{\textcolor[rgb]{0.56,0.35,0.01}{\textbf{\textit{#1}}}}
\usepackage{graphicx,grffile}
\makeatletter
\def\maxwidth{\ifdim\Gin@nat@width>\linewidth\linewidth\else\Gin@nat@width\fi}
\def\maxheight{\ifdim\Gin@nat@height>\textheight\textheight\else\Gin@nat@height\fi}
\makeatother
% Scale images if necessary, so that they will not overflow the page
% margins by default, and it is still possible to overwrite the defaults
% using explicit options in \includegraphics[width, height, ...]{}
\setkeys{Gin}{width=\maxwidth,height=\maxheight,keepaspectratio}
% Set default figure placement to htbp
\makeatletter
\def\fps@figure{htbp}
\makeatother
\setlength{\emergencystretch}{3em} % prevent overfull lines
\providecommand{\tightlist}{%
  \setlength{\itemsep}{0pt}\setlength{\parskip}{0pt}}
\setcounter{secnumdepth}{-\maxdimen} % remove section numbering

\title{Ant colony}
\author{Jan Lennartz \& Andrei Chirita}
\date{11/22/2020}

\begin{document}
\maketitle

\begin{verbatim}
## 
## Attaching package: 'igraph'
\end{verbatim}

\begin{verbatim}
## The following objects are masked from 'package:stats':
## 
##     decompose, spectrum
\end{verbatim}

\begin{verbatim}
## The following object is masked from 'package:base':
## 
##     union
\end{verbatim}

\newpage
\tableofcontents
\newpage

\hypertarget{introduction}{%
\section{Introduction}\label{introduction}}

Ant colonies have a complex and fascinating social structure that may
bring answers to a multitude of scientific questions. Usually the nest
are organized in a stratified manner with a queen at the center and
numerous workers doing tasks needed for the upkeep of the colony. The
study for which the data we worked on was collected sought to understand
the social structure of \emph{Camponotus fellah} ants, what are the
groups inside the colonies and what factors define them.

\hypertarget{the-data}{%
\subsection{The data}\label{the-data}}

The original data of the study consist of more than 9 million observed
interactions between ants collected for 41 days from ants belonging to 6
colonies. For our project we decided to work on \ldots..

\hypertarget{preliminary-data-exploration}{%
\subsubsection{Preliminary data
exploration}\label{preliminary-data-exploration}}

In the following section we will show how the data is structured for a
single colony in a single day (namely day 17, colony 1). We will start
by loading the data using the get\_graph function that we built
previously.

\begin{Shaded}
\begin{Highlighting}[]
\NormalTok{g <-}\StringTok{ }\KeywordTok{get_graph}\NormalTok{(}\DataTypeTok{colony =} \DecValTok{1}\NormalTok{, }\DataTypeTok{day =} \DecValTok{17}\NormalTok{)}
\end{Highlighting}
\end{Shaded}

Next we will have a look at the vertices of the graph, as can be seen
there are 99 of them in this graph.

\begin{Shaded}
\begin{Highlighting}[]
\KeywordTok{V}\NormalTok{(g)}
\end{Highlighting}
\end{Shaded}

\begin{verbatim}
## + 99/99 vertices, named, from 957dfe1:
##  [1] Ant621 Ant620 Ant356 Ant540 Ant115 Ant117 Ant113 Ant191 Ant190 Ant217
## [11] Ant492 Ant324 Ant552 Ant257 Ant255 Ant26  Ant27  Ant23  Ant29  Ant482
## [21] Ant137 Ant400 Ant127 Ant139 Ant560 Ant564 Ant243 Ant242 Ant50  Ant52 
## [31] Ant55  Ant142 Ant308 Ant148 Ant149 Ant387 Ant380 Ant462 Ant42  Ant43 
## [41] Ant44  Ant501 Ant507 Ant153 Ant260 Ant156 Ant159 Ant158 Ant311 Ant268
## [51] Ant390 Ant395 Ant394 Ant4   Ant7   Ant6   Ant0   Ant76  Ant73  Ant593
## [61] Ant19  Ant614 Ant599 Ant13  Ant97  Ant475 Ant169 Ant298 Ant294 Ant98 
## [71] Ant518 Ant215 Ant458 Ant218 Ant219 Ant332 Ant60  Ant63  Ant64  Ant178
## [81] Ant289 Ant173 Ant176 Ant207 Ant202 Ant209 Ant38  Ant32  Ant30  Ant530
## [91] Ant538 Ant232 Ant109 Ant238 Ant342 Ant437 Ant100 Ant347 Ant186
\end{verbatim}

Next we displayed the number of edges, as can be seen there are over
3300 edges. Thus in day 17, for the first colony there were over 3342
interactions between 99 ants.

\begin{Shaded}
\begin{Highlighting}[]
\KeywordTok{E}\NormalTok{(g)}
\end{Highlighting}
\end{Shaded}

\begin{verbatim}
## + 3342/3342 edges from 957dfe1 (vertex names):
##  [1] Ant621--Ant501 Ant621--Ant356 Ant621--Ant153 Ant621--Ant332 Ant621--Ant289
##  [6] Ant621--Ant218 Ant621--Ant156 Ant621--Ant158 Ant621--Ant311 Ant621--Ant176
## [11] Ant621--Ant294 Ant621--Ant560 Ant621--Ant207 Ant621--Ant462 Ant621--Ant115
## [16] Ant621--Ant4   Ant621--Ant117 Ant621--Ant6   Ant621--Ant507 Ant621--Ant113
## [21] Ant621--Ant76  Ant621--Ant73  Ant621--Ant38  Ant621--Ant217 Ant621--Ant32 
## [26] Ant621--Ant30  Ant621--Ant492 Ant621--Ant142 Ant621--Ant97  Ant621--Ant530
## [31] Ant621--Ant324 Ant621--Ant148 Ant621--Ant149 Ant621--Ant98  Ant621--Ant518
## [36] Ant621--Ant387 Ant621--Ant552 Ant621--Ant238 Ant621--Ant257 Ant621--Ant100
## [41] Ant621--Ant255 Ant621--Ant42  Ant621--Ant43  Ant621--Ant27  Ant621--Ant44 
## [46] Ant621--Ant23  Ant620--Ant540 Ant620--Ant115 Ant620--Ant117 Ant620--Ant113
## + ... omitted several edges
\end{verbatim}

\begin{Shaded}
\begin{Highlighting}[]
\KeywordTok{str}\NormalTok{(}\KeywordTok{vertex.attributes}\NormalTok{(g))}
\end{Highlighting}
\end{Shaded}

\begin{verbatim}
## List of 19
##  $ nb_interaction_queen   : num [1:99] NaN 8.636 0 1.091 0.0909 ...
##  $ nb_interaction_foragers: num [1:99] NaN 68.4 681.3 86.5 88.2 ...
##  $ nb_interaction_cleaners: num [1:99] NaN 169.4 65.4 126.8 143.7 ...
##  $ nb_interaction_nurses  : num [1:99] NaN 161.64 9.09 97.27 37.64 ...
##  $ visits_to_rubbishpile  : num [1:99] NaN 4.6364 0.2727 0.0909 50.2727 ...
##  $ visits_to_nest_entrance: num [1:99] NaN 0.909 166 0 0.818 ...
##  $ visits_to_brood        : num [1:99] NaN 137.909 0.636 85.909 30.909 ...
##  $ group_period4          : chr [1:99] "" " " "F" "N" ...
##  $ group_period3          : chr [1:99] "" " " "F" "N" ...
##  $ group_period2          : chr [1:99] "" "C" "F" "C" ...
##  $ group_period1          : chr [1:99] "" "N" "F" "C" ...
##  $ nb_foraging_events     : num [1:99] NaN 0 15 0 0 0 0 0 0 43 ...
##  $ age(days)              : num [1:99] NaN 57 246 372 71 190 351 71 57 344 ...
##  $ body_size              : num [1:99] NaN 141 215 149 124 ...
##  $ tag_id                 : num [1:99] NaN 620 356 540 115 117 113 191 190 217 ...
##  $ colony                 : num [1:99] NaN 4 4 4 4 4 4 4 4 4 ...
##  $ id                     : chr [1:99] "Ant621" "Ant620" "Ant356" "Ant540" ...
##  $ group                  : chr [1:99] "" "C" "F" "C" ...
##  $ name                   : chr [1:99] "Ant621" "Ant620" "Ant356" "Ant540" ...
\end{verbatim}

Each vertex has also a set of attributes like:

\begin{itemize}
\item
  Several attributes that are useful for understanding the interactions
  of the studied
\item
  Attributes that register the visits of the ant to important places of
  the colony (like the brood or the nest entrance)
\item
  The groups fitted by the authors of the study
\item
  The age of the ant (measured in days)
\item
  The body size of the ant
\end{itemize}

\hypertarget{the-original-paper}{%
\subsection{The original paper}\label{the-original-paper}}

The original paper was written by: Danielle P. Mersch, Alessandro Crespi
and Laurent Keller and explores questions related how can we separate
ant colonies into groups and what makes ants change the group they
belong to. During their study they found 3 main groups based on the
interactions between ants and concluded that age is the main factor that
determines ants to change the group they are part of. All colonies
studied had 4-years old queens and between 122 and 192 workers per
colony. Each ant was marked and followed individually and an interaction
between two ants were defined by the fact that ``the front end of one
ant was located within thetrapezoidal shape representing the other
ant''.

\hypertarget{our-questions}{%
\subsection{Our questions}\label{our-questions}}

The goal of this work is to conduct the given data set w.r.t. various
aspects. Furthermore, a validation of the key results of the original
paper is carried out. For this a sample colony is chosen on a given day.
On this network the analysis is to be done. Certain properties will be
evaluated over multiple days when required.

We will first explore the network in a descriptive manner. This includes
characteristics like degree distribution, density, diameter and more. In
the second step we have a closer look at the groups. First, we validate
that the three groups are a valid proposal for the given network. This
is done by running a clustering algorithm on the network to identify the
groups which will be compared to the labeled groupings. Second, we
investigate how frequently ants communicate within groups and compare
this to the level of communication between groups. We answer the
question how fast information can be spread in the network and compare
this to the result of the paper. Additionally, we calculate the
centrality of specific ant or groups (e.g.~the queen) w.r.t. different
measures. Furthermore, we review several properties of the ants and
their correlation with the groups (e.g.~age, size).

\end{document}
